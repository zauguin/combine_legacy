\documentclass{article}
\usepackage{iftex}
\unless\iftutex
  \usepackage[T5,T1]{fontenc}
\fi
\usepackage[vietnamese, polish, ngerman]{babel}
\usepackage{mlmodern}
\usepackage{blindtext}
\usepackage{lipsum}
\begin{document}
\pretolerance=-1
\blinddocument
\makeatletter

Ein Satz mit ß.

\textsf{\blindtext}

\texttt{\blindtext}

\setlipsum{text=lipsum-cs}
\lipsum

\begin{otherlanguage}{vietnamese}
Tiếng Việt là ngôn ngữ của người Việt và là ngôn ngữ chính thức tại Việt Nam. Đây là tiếng mẹ đẻ của khoảng 85\% dân cư Việt Nam cùng với hơn 4 triệu Việt kiều.\end{otherlanguage}\ (\textsc{Wikipedia})\footnote{Full optical size support, as in Knuth's Computer Modern. And now here's some Polish: \foreignlanguage{polish}{Język polski, polszczyzna – język z grupy zachodniosłowiańskiej (do której należą również czeski, kaszubski słowacki i języki łużyckie), stanowiącej część rodziny indoeuropejskiej. Funkcjonuje jako język urzędowy Polski oraz należy do oficjalnych języków Unii Europejskiej.} (\textsc{Wikipedia})}
\end{document}
